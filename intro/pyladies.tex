%\documentclass{beamer}
%
%\title{Introduction to debugging in Python}
%\author{PyladiesBCN - @lpmayos}
%\date{February 26, 2015}
%
%\usetheme{pyladies}
%
%\begin{document}
%\maketitle


\documentclass{beamer}
\usetheme{pyladies}

\usepackage[utf8]{inputenc}
\usepackage[T1]{fontenc}
\usepackage{graphicx}
\usepackage{listings} 

%% Use any fonts you like.
\usepackage{helvet}
\graphicspath{ {IMG/} }

\title{Django 1: Introduction to Django Models}
\subtitle{Models and tables}
\author{Elena He}
\date{\today}
\institute{PyladiesBCN}

\begin{document}

\begin{frame}[plain,t]
\titlepage
\end{frame}



\section{What are we going to do}
	\begin{frame}
	\frametitle{Summary}
	\tableofcontents
	\end{frame}

	\subsection{Manifest}
	\begin{frame}
		\frametitle{Goal}
		Who haven't heard about Django?
		\pause
		\begin{columns}
			\column{.5\textwidth}
			DO:
				\begin{itemize}[<+->]
				\item Use Django ORM
				\item Build a model
				\item Learn SQL
				\item Work with MySQL
				\end{itemize}
			\column{.5\textwidth}
			DO NOT:
				\begin{itemize}[<+->]
				\item Install Django
				\item Go deep into details of Django inner parts
				\item Hack anything
				\item Break anything (consciounsly)
				\end{itemize}
		\end{columns}
	\end{frame}
\section{Introduction}
	\subsection{What's a Model}
	\begin{frame}
		\frametitle{What's a Model}
		A model is a way to structure the information to simulate reality. \pause
		\includegraphics[width=\textwidth]{perception} 
	\end{frame}

	\subsection{Object Oriented Programming}
	\begin{frame}
		\frametitle{OOP}
		\begin{itemize}[<+->]
			\item All the domain is represented as objects.
			\item Each object is independent
			\item Each object has his responsabilities
			\item Each object has his functions, attributes
			\item Objects interact politey among their acquaintances
		\end{itemize}
	\end{frame}

	\subsection{Good Practices}
	\begin{frame}
		\frametitle{Good Practices}	
		\begin{columns}[c]
			\column{.5\textwidth} 
			\begin{itemize}[<+->]
				\item Use logic 
				\item Abstraction 
				\item Do not repeat yourself 
				\item Single responsability 
				\item KISS
			\end{itemize}
			\column{.5\textwidth} 
				\includegraphics<1>[width=\textwidth]{logical}
				\includegraphics<2>[width=\textwidth]{bull11}
				\includegraphics<3>[width=\textwidth]{repeat}
				\includegraphics<4>[width=\textwidth]{single}
				\includegraphics<5>[width=\textwidth]{kiss}
		\end{columns}
	\end{frame}

\section{DataBase Management Systems (DBMS)}

	\subsection{What's a Database}
	\begin{frame}
		\frametitle{Databases}
		A database is a black box who stores and manages the information \\ 
		\includegraphics<1>[width=.6\textwidth]{blackbox} 
		\pause
		How it stores/retrieves the information is not of our concern. \\ \pause  
		{\tt It is trustworthy}\\ \pause
		We use DB specific language to communicate with a DBMS. \\\pause 
		{\tt SQL}\\ \pause
		Django ORM is a layer of abstraction to release this communication ( developer - DBMS ) to Django. \\\pause
		You can invest your time doing something productive!
		
	\end{frame}

	\begin{frame}
	\frametitle{DBMS Stantard Organization}
		\begin{columns}
		\column{.5\textwidth}
		\begin{itemize}[<+->]
			\item DBMS is the whole system. 
			\item DBMS is divided into DATABASES
			\item Each DATABASE has his TABLES
			\item Each USER has specific permissions to access/operate any part of the DBMS
			\item ROOT can do ANYTHING! Be aware of his usage
		\end{itemize}
		\column{.5\textwidth}
		\includegraphics<1-2>[width=\textwidth]{building} 
		\includegraphics<3>[width=\textwidth]{apartamento} 
		\includegraphics<4>[width=\textwidth]{key} 
		\includegraphics<5>[width=\textwidth]{root} 
		\end{columns}
	\end{frame}

\section{SQL Syntax}

	\begin{frame}
		\frametitle{Introduction to SQL}
		SQL (Standard Query Language) is Standarized by ANSI \\\pause
		SQL is not case sensitive \\\pause
		Each DBMS implements his own version \\\pause
		{\tt Same-same, different-different} \\\pause
		Let's start with SQL Syntax !
	\end{frame}

	\subsection{Database Concerned}
	\begin{frame}
		\frametitle{Operations over Databases}
		%\begin{columns}[c]
		%	\column{.5\textwidth} 
			\begin{itemize}[<+->]
			 	\item Create Database
			 	\begin{itemize}
			 		\item CREATE DATABASE newDatabaseName
			 	\end{itemize}
			 	\item Delete Database
			 	\begin{itemize}
			 		\item DROP DATABASE databaseName
			 	\end{itemize}
			 	\item Modify Database
			 	\begin{itemize}
			 		\item ALTER DATABASE databaseName specs
			 	\end{itemize}
			\end{itemize}
	\end{frame}

	\subsection{Table Concerned}
	\begin{frame}
	 	\frametitle{Operations over Tables}
	 	\begin{itemize}[<+->]
	 		\item Create Table
	 		\begin{itemize}
	 			\item CREATE TABLE newTableName
	 		\end{itemize} 		
	 		\item Delete Table
	 		\begin{itemize}
		 		\item DROP TABLE tableName
			\end{itemize}
	 		\item Modify Table
	 		\begin{itemize}
	 			\item ALTER TABLE tableName specs
			\end{itemize}
	 	\end{itemize}
	\end{frame}

	\subsection{Item Concerned}
	\begin{frame}
		\frametitle{Operations over Items}
	 	\begin{itemize}[<+->]
	 		\item Select Item/s
	 		\begin{itemize}
		 		\item SELECT Items schemaName.tableName 
			\end{itemize}
	 		\item Update Item/s
	 		\begin{itemize}
		 		\item  UPDATE tableName SET col1=something ...
			\end{itemize}
	 		\item Insert Item/s
	 		\begin{itemize}
		 		\item INSERT INTO tableName (col1,..., coln) VALUES (11,...,1n),(...),(n1,..., nn)
			\end{itemize}
	 		\item Delete Item/s
	 		\begin{itemize}
		 		\item DELETE FROM tableName
			\end{itemize}
	 	\end{itemize}
	\end{frame}

\section{Exercises}

	\begin{frame}
	 	\frametitle{Exercises time!}
	 	\begin{itemize}
	 		\item Let's create a Django Model
	 		\item Import data to the model
	 		\item Play with the data
	 	\end{itemize}
	\end{frame}

	\subsection{Creating Django Model}
	\begin{frame}
		\frametitle{Create Model}
		\begin{itemize}[<+->]
			\item Model related to Employee
			\item 6 Model objects
			\item Reference: https://docs.djangoproject.com/en/1.8/ref/models/fields/
		\end{itemize}
	\end{frame}

	\subsection{Import data to the model}
	\begin{frame}
		\frametitle{Import data to the model}
		\begin{itemize}[<+->]
			\item clone repository from github
			\begin{itemize}
				\item git clone https://github.com/pyladies-bcn/django1.git
			\end{itemize}
			\item go to the folder django1/db or django1/exercises/2_import_data
			\item locate employees_personalized.sql
			\item import it to your mysql installation
			\item check there are no errors. Errors are evil!
		\end{itemize}
	\end{frame}

	\subsection{Play with the data}
	\begin{frame}
		\frametitle{Play with the data}
		After populating our model with registers, we have lots of data. \\ \pause
		We are going to look for  INTERESTING  Employees.
		\begin{itemize}[<+->]
			\item Look for the oldest/youngest employee
			\item Look for the employee with the highest/lowest salary
			\item Look for the most veteran/new employee
			\item The department with more employees
			\item The department with the youngest manager
			\item Anything you can imagine!
		\end{itemize}
	\end{frame}

\ThankYouFrame

\end{document}